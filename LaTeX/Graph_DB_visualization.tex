%%%%%%%%%%%%%%%%%%%%%%% file template.tex %%%%%%%%%%%%%%%%%%%%%%%%%
%
% This is a general template file for the LaTeX package SVJour3
% for Springer journals.          Springer Heidelberg 2010/09/16
%
% Copy it to a new file with a new name and use it as the basis
% for your article. Delete % signs as needed.
%
% This template includes a few options for different layouts and
% content for various journals. Please consult a previous issue of
% your journal as needed.
%
%%%%%%%%%%%%%%%%%%%%%%%%%%%%%%%%%%%%%%%%%%%%%%%%%%%%%%%%%%%%%%%%%%%
%
% First comes an example EPS file -- just ignore it and
% proceed on the \documentclass line
% your LaTeX will extract the file if required
\begin{filecontents*}{example.eps}
%!PS-Adobe-3.0 EPSF-3.0
%%BoundingBox: 19 19 221 221
%%CreationDate: Mon Sep 29 1997
%%Creator: programmed by hand (JK)
%%EndComments
gsave
newpath
  20 20 moveto
  20 220 lineto
  220 220 lineto
  220 20 lineto
closepath
2 setlinewidth
gsave
  .4 setgray fill
grestore
stroke
grestore
\end{filecontents*}
%
\RequirePackage{fix-cm}
%
%\documentclass{svjour3}                     % onecolumn (standard format)
%\documentclass[smallcondensed]{svjour3}     % onecolumn (ditto)
\documentclass[smallextended]{svjour3}       % onecolumn (second format)
%\documentclass[twocolumn]{svjour3}          % twocolumn
%
\smartqed  % flush right qed marks, e.g. at end of proof
%
\usepackage{graphicx}
\usepackage{todonotes}
%
% \usepackage{mathptmx}      % use Times fonts if available on your TeX system
%
% insert here the call for the packages your document requires
%\usepackage{latexsym}
% etc.
%
% please place your own definitions here and don't use \def but
% \newcommand{}{}
%
% Insert the name of "your journal" with
% \journalname{myjournal}
%
\begin{document}

\title{Silico-Paleontogy%\thanks{Grants or other notes
%about the article that should go on the front page should be
%placed here. General acknowledgments should be placed at the end of the article.}
}
\subtitle{Visualizing genetic programming ancestries with graph databases}
% \subtitle{Digging through the relics of artificial evolution}

%\titlerunning{Short form of title}        % if too long for running head

\author{Nicholas Freitag McPhee         \and
        Thomas Helmuth \and
        Lee Spector \and
        Maggie Casale \and
        Mitch Finzel \and
        Any Others? %etc.
}

%\authorrunning{Short form of author list} % if too long for running head

\institute{N. McPhee \at
              Division of Science and Mathematics \\
              University of Minnesota, Morris \\
              Morris, MN USA \\
              Tel.: +320-589-6320\\
              \email{mcphee@moris.umn.edu}           %  \\
%             \emph{Present address:} of F. Author  %  if needed
           \and
           S. Author \at
              second address
}

\date{Received: date / Accepted: date}
% The correct dates will be entered by the editor


\maketitle

\begin{abstract}
	Much evolutionary computation research collects and reports thin statistical
	summaries that fail to capture or convey the complex dynamics of our
	actual runs, and all their micro-level events. In principle, however, we control
	every moment in our evolutionary runs and could record all these events.
	This would, for example, allow us to build full ancestry trees, identifying and analyzing the key events that led to success in a run, or blocked it. Here we
	illustrate the use of graph databases to collect rich details from genetic programming runs, and then describe information rich visualizations designed to aid in the understanding of these large, complex data sets.
	These visualizations have been key in discovering surprising and significant
	events in our runs, leading to new understanding of the dynamics of our
	evolutionary systems.
\keywords{Genetic programming \and Visualization \and Ancestries \and Graph databases}
% \PACS{PACS code1 \and PACS code2 \and more}
% \subclass{MSC code1 \and MSC code2 \and more}
\end{abstract}

\section*{Outline}
\marginpar{Obviously this has to go away.}

My general idea for the outline:
\begin{enumerate}
	\item Introduction: Rant about throwing away all that data and talk about why that's a bummer. This will actually probably be pretty long, maybe long enough that it should be a separate section?
	\item Graph Databases: Describe the basic idea, and our particular (current) data schema (include the semantics).
	\item Visualizations: Describe different edges, boxes sizes and shapes, colors (even though they won't display here)
	\begin{enumerate}
		\item Full runs
		\item Just ancestors
		\item Filtered by genes
	\end{enumerate}
	\item What have we learned? Hyperselection. Semantics hyperselection. Use the end of the poster run to illustrate both of these features.
	\item Conclusions: Graph DBs make this work possible and more people should do it. Does suck up a lot of disk space and takes a while, though, so there's work to be done. Need to get better at automating processes and combining analysis across multiple runs.
\end{enumerate}

\section{Introduction}
\label{sec:intro}

\begin{quote}
	In order to discover the actual steps by which the male of any existing 
	bird has acquired his magnificent colours or other ornaments, we ought 
	to behold the long line of his extinct progenitors; \emph{but this is obviously impossible}.\footnote{Emphasis ours.} \\
	\\
	--- Charles Darwin, in reference to peacocks~\cite{Darwin} 
\end{quote}

After people began to realize that species were not fixed, and that the fossil 
record contained evidence of a long and complex natural history, it also
became clear that this record was terribly incomplete. In the quote above
Darwin clearly realized the potential value of a complete ancestral record in
understanding evolutionary processes, but he also took it as a given that
such a record was not to be had.

The incompleteness of the historical record, both human and natural, continues
to challenge scholars in a host of fields. Paleontologists work with a 
profoundly
incomplete fossil record, often describing and classifying entire categories of
organisms based a few teeth. The human historical record is no more complete,
with numerous valuable resources lost to the ravages of time. Worse, these 
gaps are often neither random nor symmetric, substantially skewing our view 
of that history. Teeth, for example, are often all we have of ancient 
sharks and their kin because their skeletons are made of cartilage, which
typically doesn't fossilize, and the lives and ideas of people with wealth, 
power, and education are far more likely to be preserved than those of the
poor and illiterate.

While it was impossible for Darwin to ``behold the long line of \ldots 
[peacock] progenitors'', in evolutionary computation it is in principle possible
to save \emph{all} the data from a run, and indeed explore all the progenitors
of a successful individual. As described in Section~\ref{sec:EC_pattern},
however, we rarely collect and analyze this type of data, instead only sharing
thin statistically summaries, pale shadows of the extremely complex 
processes that stand behind our research. 
In this paper, however, we propose
saving far more data tha has been common in the field, recording all the ``little'' low-level events that ultimately drive whatever large-scale trends 
we observe in our runs. While the majority of EC research fails to record more
than a tiny fraction of this detail, there are exceptions, some of which 
will be discussed in Section~\ref{sec:related}.

Recording all these low-level events does represent a substantial increase 
in the amount of data
collected, which means that we need tools and techniques to store, explore,
analyze, and share results from that data. In recent work~\cite{graph_db_work}
we have found graph databases (discussed in
Section~\ref{sec:graph_DBs}) to be an effective tool for managing the 
substantial collection of data generated during a run. In order to understand
the particulars of what we're storing and why, it will be necessary to provide
a brief introduction to the the Push programming language and
PushGP~\cite{PushGP} in Section~\ref{sec:Push}.

Visualizations are central to the success
of our work with graph databases, and Section~\ref{sec:visualizations}
describes our design of data rich 
visualizations that allow us to effectively explore and share results such as
complex ancestry graphs. We will then share examples of things we've learned
from such ancestry graphs in Section~\ref{sec:learned}, and present conclusions
and ideas for future work in Section~\ref{sec:conclusions}.

\section{A common pattern in EC research}
\label{sec:EC_pattern}

A common pattern in published evolutionary computation research is to
compare approach $A$ and approach $B$ (perhaps two different selection
mechanisms, or two different recombination operators) by doing a number
of runs with each approach, often on several different ``representative''
problems. Then a table of summary statistics is reported that (usually)
suggests that one of the two methods had some sort of advantage over
the other.

As an example, Table~\ref{tab:example_table} (which is a subset of a 
much larger dataset presented in \cite{Helmuth:Benchmarks}) presents
the number of successes out of 100 independent when comparing three
different selection mechanisms across a number of software synthesis
problems. The results in this table (which are supported and extended
in the full table) suggest that on this set of test problems lexicase
selection is generally the most successful.

\begin{table}
	% table caption is above the table
	\caption{An example of summary reseaerch results, taken from
	\cite{Helmuth:Benchmarks}. This shows the number of successes
    out of 100 independent runs on several software sythesis problems,
    for each of three selection mechanisms: Tournament, Implicit Fitness Sharing (IFS), and Lexicase. \underline{Underline} indicates a statistically
    significant advantage over both other selection methods at $p < 0.05$
    based on a pairwise chi-square test with Holm correction.}
	\label{tab:example_table}       % Give a unique label
	% For LaTeX tables use
	\begin{tabular}{llll}
		\hline\noalign{\smallskip}
		Problem & Tourn. & IFS & Lex. \\
		\noalign{\smallskip}\hline\noalign{\smallskip}
		Number IO & 68 & 72 & \underline{98} \\
		Smallest & 75 & \underline{98} & 81 \\
		String Lengths Backwards & 7 & 10 & \underline{66} \\
		Replace Space With Newline & 8 & 16 & \underline{51} \\
		Vector Average & 14 & 13 & 16 \\
		Small Or Large & 3 & 3 & 5 \\
		For Loop Index & 0 & 0 & 1 \\
		String Differences & 0 & 0 & 0 \\
		\noalign{\smallskip}\hline
	\end{tabular}
\end{table}

This is a useful result, and clearly tells us something valuable about
these three selection mechanisms in the tested domain. It also leaves
so much unsaid, however, about the \emph{why} of this result. It's great
to know that lexicase selection has a distinct and general advantage on 
these problems, but we have no idea \emph{why} lexicase selection is
outperforming the other two approaches. The handful of numbers in
Table~\ref{tab:example_table} are summarizing the complex ``lives'' 
and interactions of 100's of millions of individuals, but those numbers
tell us nothing about how those billions of micro-events add up to this
overall result. This is not simply a pedantic observation, as this lost 
information also represents a lost opportunity for greater understanding. 

We might, for example, see a similar table in the social sciences, 
perhaps listing economic metrics like gross domestic product for 
various countries. That might give us a high level view of the relationship
of those countries economies, but no one would claim to \emph{understand}
those relationships based on the table alone. Understanding would come
from unpacking that small set of numbers, and trying to describe the
mechanisms that underlie them. 

Yet published research in evolutionary
computation regularly ``ends'' with something like 
Table~\ref{tab:example_table}, with no empirically grounded discussion 
of the mechanisms that might lead to these summary results. The detailed
data about the events that drive these results aren't presented or discussed,
and likely aren't collected or even available to the researchers themselves, 
as it is extremely common in evolutionary computation research 
to effectively throw away all that data as the runs progress.

\todo[inline]{Skim through GECCO papers counting number that have
	this form?}

\begin{figure}
	\includegraphics[width=\linewidth]{Figures/rswn_diversity}
	\caption{The diversity of error vectors over time for 100 runs each
	on Replace Space With Newline using lexicase selection (top) and 
    tournament selection (bottom). Taken from~\cite{diversity_gecco}.}
	\todo[inline]{We should get a ``cleaner'' version of this with better labels.}
	\todo[inline]{Need to get the correct citation for this.}
	\label{fig:rswndiversity}
\end{figure}

No all research ends with just a table, however, and many published papers
do contain data (often in the forms of visualization) that capture
significant aspects of run dynamics. Figure~\ref{fig:rswndiversity}, for
example, shows the change in diversity of populations over time using
two different selection mechanisms. We can see that lexicase
selection generally maintains a much higher level of diversity than
than tournament selection, suggesting a possible explanation for the
improved performance seen in Table~\ref{tab:example_table}.

\begin{figure}
	\includegraphics[width=0.7\linewidth]{Figures/run6_lexicase_rswn_diversity}
	\caption{The diversity of error vectors over time for a single run of
	Replace Space With Newline using lexicase selection. Taken from~\cite{diversity_gecco}.}
	\todo[inline]{We should get a ``cleaner'' version of this with better labels.}
	\todo[inline]{Need to get the correct citation for this.}
	\label{fig:run6lexicaserswndiversity}
\end{figure}

Yet while Figure~\ref{fig:rswndiversity} conveys \emph{far} more data
than Table~\ref{tab:example_table}, it still blurs potentially important
details of individual runs. Figure~\ref{fig:run6lexicaserswndiversity} shows
the change in diversity over a single run using lexicase selection. Here we 
see three substantial changes in diversity: a strong drop and rapid recovery
at the very beginning of the run, a substantial dip and rise in middle of the run, and a steep drop right at the end of the run. All of these events seem
significant and deserving of explanation. If this were a graph of any meaningful data in the social sciences, for example, these periods of
pronounced change would almost certainly be the first things to be
examined, and likely whole theories would be developed around these
phenomena. While interesting dynamics such as this are probably quite
common in evolutionary computation runs, we rarely see them, and they
are rarely discussed or explored. In general, unfortunately, they
\emph{can't} be explored, because it is rarely the case that we save the
kinds of data we'd need to be able to go back and find out what happened
at these interesting moments.

Here we propose to save detailed event-level data from a run:
\begin{itemize}
	\item Who was selected to be a parent
	\item What mutation or recombination operator was applied
	\item What child resulted
	\item Exactly which pieces of genetic material (``genes'') were
	modified and transferred, and in what positions
\end{itemize}
In principle, if we can save this data we can go back to key moments
in runs (like the drops in diversity in 
Figure~\ref{fig:run6lexicaserswndiversity}) and reconstruct the specific
events that led to our initial macro-level observations.

There are, however, several potential obstacles to this approach: storing, 
processing, and understanding the data. In the early days of genetic programming, when a gigabyte
of disk space cost thousands of dollars, saving this level of data for dozens 
or hundreds of runs simply wasn't feasible for many researchers. The
profound changes in the cost and capacity of storage has, however, 
largely made this a non-issue.

\todo[inline]{I found myself talking about genes here, but we haven't
described Push, etc. I don't really want to do that before this sections,
as this is the key motivation, so we just need to remove any reference
to genes and stick to individuals (which everyone presumably knows
about)?}

Simply being able to store all the data isn't sufficient, however; we need
tools that allow us to effectively work with the data. Recent developments
in graph databases have largely resolved this issue as well, as described
in Section~\ref{sec:graph_DBs}. For the moment it's sufficient to understand
that graph databases allow us to store and easily query all this data. We can,
for example, easily determine \emph{all} of the ancestors of a given
individual, such as a successful ``winner'' that completely solves the problem.
Or we can filter that result by looking only at ancestors that contributed
at least one gene to the target individual.

Lastly is the problem of understanding the data, where we have found
visualization to be a vital technique. Single runs regularly contain hundreds
of thousands or millions of individuals, each of which typically contains
hundreds of genes. High quality visualizations have been critical to 
Understanding and sharing such large data sets; these will be described 
in more detail in Section~\ref{sec:visualizations}.

\section{Related work}
\label{sec:related}

We are not aware of any other work in evolutionary computation using graph
databases to store and analyze full ancestry trees in a systematic way. That
said, there is other work where ancestry information has been collected,
analyzed, and visualized.

\todo[inline]{Crib stuff from, e.g., the VizGEC paper in Denver.}

\section{PushGP}
\label{sec:Push}

\todo[inline]{Someone want to fill this in? We need to know about the basic
linear structure of genomes, the idea of genes, and presumably a little
about getting from Plush to Push and how Push programs work.}

\section{Graph databases}
\label{sec:graph_DBs}

Instead of storing data in tables like relational databases, graph 
databases\footnote{We have had success with both Neo4J and Titan DB.} 
store data as nodes and edges in a graph, with the ability to store additional
data on both nodes and edges much like a document-oriented database.
This is a natural fit for storing and analyzing ancestry data from evolutionary
computation runs, as a key type of data we would like to store and explore
are parent-child relationships, along with information like where genes came
from, all of which have straightforward representations as edges between
nodes.

\section{Visualizations}
\label{sec:visualizations}

\section{What have we learned?}
\label{sec:learned}

\section{Conclusions}
\label{sec:conclusions}

\todo[inline]{We may want a separate future work section? Not sure.}

\section*{Random examples from Springer}

\todo[inline]{This is all stuff from Springer and needs to be removed.}

Text with citations \cite{RefB} and \cite{RefJ}.

\subsection{Subsection title}
\label{sec:2}
as required. Don't forget to give each section
and subsection a unique label (see Sect.~\ref{sec:1}).
\paragraph{Paragraph headings} Use paragraph headings as needed.
\begin{equation}
a^2+b^2=c^2
\end{equation}

% For one-column wide figures use
\begin{figure}
% Use the relevant command to insert your figure file.
% For example, with the graphicx package use
  \includegraphics{example.eps}
% figure caption is below the figure
\caption{Please write your figure caption here}
\label{fig:1}       % Give a unique label
\end{figure}
%
% For two-column wide figures use
\begin{figure*}
% Use the relevant command to insert your figure file.
% For example, with the graphicx package use
  \includegraphics[width=0.75\textwidth]{example.eps}
% figure caption is below the figure
\caption{Please write your figure caption here}
\label{fig:2}       % Give a unique label
\end{figure*}
%
% For tables use
\begin{table}
% table caption is above the table
\caption{Please write your table caption here}
\label{tab:1}       % Give a unique label
% For LaTeX tables use
\begin{tabular}{lll}
\hline\noalign{\smallskip}
first & second & third  \\
\noalign{\smallskip}\hline\noalign{\smallskip}
number & number & number \\
number & number & number \\
\noalign{\smallskip}\hline
\end{tabular}
\end{table}


%\begin{acknowledgements}
%If you'd like to thank anyone, place your comments here
%and remove the percent signs.
%\end{acknowledgements}

% BibTeX users please use one of
\bibliographystyle{spbasic}      % basic style, author-year citations
%\bibliographystyle{spmpsci}      % mathematics and physical sciences
%\bibliographystyle{spphys}       % APS-like style for physics
\bibliography{GPEM_viz.bib}   % name your BibTeX data base

\end{document}
% end of file template.tex

